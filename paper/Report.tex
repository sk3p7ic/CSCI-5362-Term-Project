\documentclass[conference]{IEEEtran}
\IEEEoverridecommandlockouts
\usepackage{amsmath,amssymb,amsfonts}
\usepackage{algorithmic}
\usepackage{graphicx}
\usepackage{textcomp}
\usepackage{xcolor}
\def\BibTeX{{\rm B\kern-.05em{\sc i\kern-.025em b}\kern-.08em
    T\kern-.1667em\lower.7ex\hbox{E}\kern-.125emX}}
\begin{document}

%\title{How Safe is C to Rust Translation Using LLMs}
\title{How Safe is C to Rust Translation Using LLMs?}

\author{
\IEEEauthorblockN{Kenneth Fulton}
\IEEEauthorblockA{\textit{Dept. of CEMS} \\
\textit{Texas A\&M University - San Antonio}\\
San Antonio, US \\
kfult01@jaguar.tamu.edu}
\and
\IEEEauthorblockN{Joshua Ibrom}
\IEEEauthorblockA{\textit{Dept. of CEMS} \\
\textit{Texas A\&M University - San Antonio}\\
San Antonio, US \\
jibro01@jaguar.tamu.edu}
}

\maketitle

\begin{abstract}
Lorem ipsum.
\end{abstract}

\begin{IEEEkeywords}
programming language translation, ai safety
\end{IEEEkeywords}

\section{Background}
Traditional systems-level programming of embedded software, operating systems,
and other low-level applications has been done with language such as C so that
memory and other system components may be managed by the programmer as needed.
With this form of programming also comes the potential for pitfalls that leave
applications ``open to exploits, crashes, or corruption'' 
\cite{klabnik2018rust_book}. With the increasing popularity of Rust, a
systems-level programming language promising safety from the common C / C++
pitfalls, there is now the question of how we might perform some re-write of
critical system software from their potentially insecure C versions to ones
written in Rust \cite{klabnik2018rust_book, emre2021translating}.

One such method of performing a re-write of some codebase is to attempt to
hand-roll a new version of that software by manually translating line-by-line
given C code to Rust code---a method that is time consuming, expensive, and
may lead to there being missing or incorrectly-implemented blocks of code.
Another such approach that has now become far more feasible with the
prevalence and advanced reasoning abilities of \textit{large language models}
(LLMs) is to use those LLMs to perform automatic translation with human
verification.

\subsection{Research Questions}
\begin{enumerate}
    \item How reliably can LLMs resolve memory safety errors when converting 
        C code to Rust?
    \item How efficiently can LLMs translate C code to Rust?
    \item How can you prove that Rust code is safe?
\end{enumerate}


\section{Related Work}

\section{Proposed Approach}

\section{Experimentation}

\section{Experiment Outcomes and Observations}

\section{Conclusion}

\bibliographystyle{IEEEtran}
\bibliography{paper/sources}
\end{document}